\section{Type system and registration
process}\label{type-system-and-registration-process}

GObject is a base object. We don't usually use GObject itself. Because
GObject is very simple and not enough to be used by itself in most
situations. Instead, we use descendant objects of GObject such as many
kinds of GtkWidget. We can rather say such derivability is the most
important feature of GObject.

This section describes how to define a child object of GObject.

\subsection{Name convention}\label{name-convention}

An example of this section is an object represents a real number. It is
not so useful because we have already had double type in C language to
represent real numbers. However, I think this example is not so bad to
know the technique how to define a child object.

First, you need to know the naming convention. An object name consists
of name space and name. For example, ``GObject'' consists of a name
space ``G'' and a name ``Object''. ``GtkWidget'' consists of a name
space ``Gtk'' and a name ``Widget''. Let the name space be ``T'' and the
name be ``Double'' of the new object. In this tutorial, we use ``T'' as
a name space for all the objects we make.

TDouble is the object name. It is a child object of GObject. It
represents a real number and the type of the number is double. It has
some useful functions.

\subsection{Define TDoubleClass and
TDouble}\label{define-tdoubleclass-and-tdouble}

When we say ``type'', it can be the type in the type system or C
language type. For example, GObject is a type name in the type system.
And char, int or double is C language types. When the meaning of the
word ``type'' is clear in the context, we just call it ``type''. But if
it's ambiguous, we call it ``C type'' or ``type in the type system''.

TDouble object has the class and instance. The C type of the class is
TDoubleClass. Its structure is like this:

\begin{lstlisting}[language=C]
typedef struct _TDoubleClass TDoubleClass;
struct _TDoubleClass {
  GObjectClass parent_class;
};
\end{lstlisting}

\_TDoubleClass is a C structure tag name and TDoubleClass is ``struct
\_TDoubleClass''. TDoubleClass is a newly created C type.

\begin{itemize}
\tightlist
\item
  Use typedef to define a class type.
\item
  The first member of the structure must be the parent's class
  structure.
\end{itemize}

TDoubleClass doesn't need its own member.

The C type of the instance of TDouble is TDouble.

\begin{lstlisting}[language=C]
typedef struct _TDouble TDouble;
struct _TDouble {
  GObject parent;
  double value;
};
\end{lstlisting}

This is similar to the structure of the class.

\begin{itemize}
\tightlist
\item
  Use typedef to define an instance type.
\item
  The first member of the structure must be the parent's instance
  structure.
\end{itemize}

TDouble has its own member, ``value''. It is the value of TDouble
instance.

The coding convention above needs to be kept all the time.

\subsection{Creation process of a child of
GObject}\label{creation-process-of-a-child-of-gobject}

The creation process of TDouble type is similar to the one of GObject.

\begin{enumerate}
\def\labelenumi{\arabic{enumi}.}
\tightlist
\item
  Registers TDouble type to the type system.
\item
  The type system allocates memory for TDoubleClass and TDouble.
\item
  Initializes TDoubleClass.
\item
  Initializes TDouble.
\end{enumerate}

\subsection{Registration}\label{registration}

Usually registration is done by convenient macro such as
\passthrough{\lstinline!G\_DECLARE\_FINAL\_TYPE!} and
\passthrough{\lstinline!G\_DEFINE\_TYPE!}. You can use
\passthrough{\lstinline!G\_DEFINE\_FINAL\_TYPE!} for a final type class
instead of \passthrough{\lstinline!G\_DEFINE\_TYPE!} since GLib version
2.70. So you don't need to care about registration details. But, in this
tutorial, it is important to understand GObject type system, so I want
to show you the registration without macro, first.

There are two kinds of types, static and dynamic. Static type doesn't
destroy its class even after all the instances have been destroyed.
Dynamic type destroys its class when the last instance has been
destroyed. The type of GObject is static and its descendant objects'
type is also static. The function
\passthrough{\lstinline!g\_type\_register\_static!} registers a type of
a static object. The following code is extracted from
\passthrough{\lstinline!gtype.h!} in the Glib source files.

\begin{lstlisting}[language=C]
GType
g_type_register_static (GType           parent_type,
                        const gchar     *type_name,
                        const GTypeInfo *info,
                        GTypeFlags      flags);
\end{lstlisting}

The parameters above are:

\begin{itemize}
\tightlist
\item
  parent\_type: Parent type.
\item
  type\_name: The name of the type. For example, ``TDouble''.
\item
  info: Information of the type. \passthrough{\lstinline!GTypeInfo!}
  structure will be explained below.
\item
  flags: Flag. If the type is abstract type or abstract value type, then
  set their flag. Otherwise, set it to zero.
\end{itemize}

Because the type system maintains the parent-child relationship of the
type, \passthrough{\lstinline!g\_type\_register\_static!} has a parent
type parameter. And the type system also keeps the information of the
type. After the registration,
\passthrough{\lstinline!g\_type\_register\_static!} returns the type of
the new object.

\passthrough{\lstinline!GTypeInfo!} structure is defined as follows.

\begin{lstlisting}[language=C]
typedef struct _GTypeInfo  GTypeInfo;

struct _GTypeInfo
{
  /* interface types, classed types, instantiated types */
  guint16                class_size;

  GBaseInitFunc          base_init;
  GBaseFinalizeFunc      base_finalize;

  /* interface types, classed types, instantiated types */
  GClassInitFunc         class_init;
  GClassFinalizeFunc     class_finalize;
  gconstpointer          class_data;

  /* instantiated types */
  guint16                instance_size;
  guint16                n_preallocs;
  GInstanceInitFunc      instance_init;

  /* value handling */
  const GTypeValueTable  *value_table;
};
\end{lstlisting}

This structure needs to be created before the registration.

\begin{itemize}
\tightlist
\item
  class\_size: The size of the class. For example, TDouble's class size
  is \passthrough{\lstinline!sizeof (TDoubleClass)!}.
\item
  base\_init, base\_finalize: These functions initialize/finalize the
  dynamic members of the class. In many cases, they aren't necessary,
  and are assigned NULL. For further information, see
  \href{https://docs.gtk.org/gobject/callback.BaseInitFunc.html}{GObject
  API Reference -- BaseInitFunc} and
  \href{https://docs.gtk.org/gobject/callback.ClassInitFunc.html}{GObject
  API Reference -- ClassInitFunc}.
\item
  class\_init: Initializes static members of the class. Assign your
  class initialization function to \passthrough{\lstinline!class\_init!}
  member. By convention, the name is
  \passthrough{\lstinline!<name space>\_<name>\_class\_init!}, for
  example, \passthrough{\lstinline!t\_double\_class\_init!}.
\item
  class\_finalize: Finalizes the class. Because descendant type of
  GObjec is static, it doesn't have a finalize function. Assign NULL to
  \passthrough{\lstinline!class\_finalize!} member.
\item
  class\_data: User-supplied data passed to the class init/finalize
  functions. Usually NULL is assigned.
\item
  instance\_size: The size of the instance. For example, TDouble's
  instance size is \passthrough{\lstinline!sizeof (TDouble)!}.
\item
  n\_preallocs: This is ignored. it has been used by the old version of
  Glib.
\item
  instance\_init: Initializes instance members. Assign your instance
  initialization function to \passthrough{\lstinline!instance\_init!}
  member. By convention, the name is
  \passthrough{\lstinline!<name space>\_<name>\_init!}, for example,
  \passthrough{\lstinline!t\_double\_init!}.
\item
  value\_table: This is usually only useful for fundamental types. If
  the type is descendant of GObject, assign NULL.
\end{itemize}

These information is kept by the type system and used when the object is
created or destroyed. Class\_size and instance\_size are used to
allocate memory for the class and instance. Class\_init and
instance\_init functions are called when class or instance is
initialized.

The C program \passthrough{\lstinline!example3.c!} shows how to use
\passthrough{\lstinline!g\_type\_register\_static!}.

\begin{lstlisting}[language=C, numbers=left]
#include <glib-object.h>

#define T_TYPE_DOUBLE  (t_double_get_type ())

typedef struct _TDouble TDouble;
struct _TDouble {
  GObject parent;
  double value;
};

typedef struct _TDoubleClass TDoubleClass;
struct _TDoubleClass {
  GObjectClass parent_class;
};

static void
t_double_class_init (TDoubleClass *class) {
}

static void
t_double_init (TDouble *self) {
}

GType
t_double_get_type (void) {
  static GType type = 0;
  GTypeInfo info;

  if (type == 0) {
    info.class_size = sizeof (TDoubleClass);
    info.base_init = NULL;
    info.base_finalize = NULL;
    info.class_init = (GClassInitFunc)  t_double_class_init;
    info.class_finalize = NULL;
    info.class_data = NULL;
    info.instance_size = sizeof (TDouble);
    info.n_preallocs = 0;
    info.instance_init = (GInstanceInitFunc)  t_double_init;
    info.value_table = NULL;
    type = g_type_register_static (G_TYPE_OBJECT, "TDouble", &info, 0);
  }
  return type;
}

int
main (int argc, char **argv) {
  GType dtype;
  TDouble *d;

  dtype = t_double_get_type (); /* or dtype = T_TYPE_DOUBLE */
  if (dtype)
    g_print ("Registration was a success. The type is %lx.\n", dtype);
  else
    g_print ("Registration failed.\n");

  d = g_object_new (T_TYPE_DOUBLE, NULL);
  if (d)
    g_print ("Instantiation was a success. The instance address is %p.\n", d);
  else
    g_print ("Instantiation failed.\n");
  g_object_unref (d); /* Releases the object d. */

  return 0;
}
\end{lstlisting}

\begin{itemize}
\tightlist
\item
  16-22: A class initialization function and an instance initialization
  function. The argument \passthrough{\lstinline!class!} points the
  class structure and the argument \passthrough{\lstinline!self!} points
  the instance structure. They do nothing here but they are necessary
  for the registration.
\item
  24-43: \passthrough{\lstinline!t\_double\_get\_type!} function. This
  function returns the type of the TDouble object. The name of a
  function is always
  \passthrough{\lstinline!<name space>\_<name>\_get\_type!}. And a macro
  \passthrough{\lstinline!<NAME\_SPACE>\_TYPE\_<NAME>!} (all characters
  are upper case) is replaced by this function. Look at line 3.
  \passthrough{\lstinline!T\_TYPE\_DOUBLE!} is a macro replaced by
  \passthrough{\lstinline!t\_double\_get\_type ()!}. This function has a
  static variable \passthrough{\lstinline!type!} to keep the type of the
  object. At the first call of this function,
  \passthrough{\lstinline!type!} is zero. Then it calls
  \passthrough{\lstinline!g\_type\_register\_static!} to register the
  object to the type system. At the second or subsequent call, the
  function just returns \passthrough{\lstinline!type!}, because the
  static variable \passthrough{\lstinline!type!} has been assigned
  non-zero value by \passthrough{\lstinline!g\_type\_register\_static!}
  and it keeps the value.
\item
  30-40 : Sets \passthrough{\lstinline!info!} structure and calls
  \passthrough{\lstinline!g\_type\_register\_static!}.
\item
  45-64: Main function. Gets the type of TDouble object and displays it.
  The function \passthrough{\lstinline!g\_object\_new!} is used to
  instantiate the object. The GObject API reference says that the
  function returns a pointer to a GObject instance but it actually
  returns a gpointer. Gpointer is the same as
  \passthrough{\lstinline!void *!} and it can be assigned to a pointer
  that points any type. So, the statement
  \passthrough{\lstinline!d = g\_object\_new (T\_TYPE\_DOUBLE, NULL);!}
  is correct. If the function \passthrough{\lstinline!g\_object\_new!}
  returned \passthrough{\lstinline!GObject *!}, it would be necessary to
  cast the returned pointer. After the creation, it shows the address of
  the instance. Finally, the instance is released and destroyed with the
  function \passthrough{\lstinline!g\_object\_unref!}.
\end{itemize}

\passthrough{\lstinline!example3.c!} is in the src/misc directory.

Execute it.

\begin{lstlisting}
$ cd src/misc; _build/example3
Registration was a success. The type is 56414f164880.
Instantiation was a success. The instance address is 0x56414f167010.
\end{lstlisting}

\subsection{G\_DEFINE\_TYPE macro}\label{g_define_type-macro}

The registration above is always done with the same algorithm.
Therefore, it can be defined as a macro such as
\passthrough{\lstinline!G\_DEFINE\_TYPE!}.

\passthrough{\lstinline!G\_DEFINE\_TYPE!} does the following:

\begin{itemize}
\tightlist
\item
  Declares a class initialization function. Its name is
  \passthrough{\lstinline!<name space>\_<name>\_class\_init!}. For
  example, if the object name is \passthrough{\lstinline!TDouble!}, it
  is \passthrough{\lstinline!t\_double\_class\_init!}. This is a
  declaration, not a definition. You need to define it.
\item
  Declares a instance initialization function. Its name is
  \passthrough{\lstinline!<name space>\_<name>\_init!}. For example, if
  the object name is \passthrough{\lstinline!TDouble!}, it is
  \passthrough{\lstinline!t\_double\_init!}. This is a declaration, not
  a definition. You need to define it.
\item
  Defines a static variable pointing to the parent class. Its name is
  \passthrough{\lstinline!<name space>\_<name>\_parent\_class!}. For
  example, if the object name is \passthrough{\lstinline!TDouble!}, it
  is \passthrough{\lstinline!t\_double\_parent\_class!}.
\item
  Defines a \passthrough{\lstinline!<name space>\_<name>\_get\_type ()!}
  function. For example, if the object name is
  \passthrough{\lstinline!TDouble!}, it is
  \passthrough{\lstinline!t\_double\_get\_type!}. The registration is
  done in this function like the previous subsection.
\end{itemize}

Using this macro reduces lines of the program. See the following sample
\passthrough{\lstinline!example4.c!} which works the same as
\passthrough{\lstinline!example3.c!}.

\begin{lstlisting}[language=C, numbers=left]
#include <glib-object.h>

#define T_TYPE_DOUBLE  (t_double_get_type ())

typedef struct _TDouble TDouble;
struct _TDouble {
  GObject parent;
  double value;
};

typedef struct _TDoubleClass TDoubleClass;
struct _TDoubleClass {
  GObjectClass parent_class;
};

G_DEFINE_TYPE (TDouble, t_double, G_TYPE_OBJECT)

static void
t_double_class_init (TDoubleClass *class) {
}

static void
t_double_init (TDouble *self) {
}

int
main (int argc, char **argv) {
  GType dtype;
  TDouble *d;

  dtype = t_double_get_type (); /* or dtype = T_TYPE_DOUBLE */
  if (dtype)
    g_print ("Registration was a success. The type is %lx.\n", dtype);
  else
    g_print ("Registration failed.\n");

  d = g_object_new (T_TYPE_DOUBLE, NULL);
  if (d)
    g_print ("Instantiation was a success. The instance address is %p.\n", d);
  else
    g_print ("Instantiation failed.\n");
  g_object_unref (d);

  return 0;
}
\end{lstlisting}

Thanks to \passthrough{\lstinline!G\_DEFINE\_TYPE!}, we are freed from
writing bothersome code like \passthrough{\lstinline!GTypeInfo!} and
\passthrough{\lstinline!g\_type\_register\_static!}. One important thing
to be careful is to follow the convention of the naming of init
functions.

Execute it.

\begin{lstlisting}
$ cd src/misc; _build/example4
Registration was a success. The type is 564b4ff708a0.
Instantiation was a success. The instance address is 0x564b4ff71400.
\end{lstlisting}

You can use \passthrough{\lstinline!G\_DEFINE\_FINAL\_TYPE!} instead of
\passthrough{\lstinline!G\_DEFINE\_TYPE!} for final type classes since
GLib version 2.70.

\subsection{G\_DECLARE\_FINAL\_TYPE
macro}\label{g_declare_final_type-macro}

Another useful macro is
\passthrough{\lstinline!G\_DECLARE\_FINAL\_TYPE!} macro. This macro can
be used for a final type. A final type doesn't have any children. If a
type has children, it is a derivable type. If you want to define a
derivable type object, use
\passthrough{\lstinline!G\_DECLARE\_DERIVABLE\_TYPE!} instead. However,
you probably want to write final type objects in most cases.

\passthrough{\lstinline!G\_DECLARE\_FINAL\_TYPE!} does the following:

\begin{itemize}
\tightlist
\item
  Declares \passthrough{\lstinline!<name space>\_<name>\_get\_type ()!}
  function. This is only declaration. You need to define it. But you can
  use \passthrough{\lstinline!G\_DEFINE\_TYPE!}, its expansion includes
  the definition of the function. So, you actually don't need to write
  the definition by yourself.
\item
  The C type of the object is defined as a typedef of structure. For
  example, if the object name is \passthrough{\lstinline!TDouble!}, then
  \passthrough{\lstinline!typedef struct \_TDouble TDouble!} is included
  in the expansion. But you need to define the structure
  \passthrough{\lstinline!struct \_TDouble!} by yourself before
  \passthrough{\lstinline!G\_DEFINE\_TYPE!}.
\item
  \passthrough{\lstinline!<NAME SPACE>\_<NAME>!} macro is defined. For
  example, if the object is \passthrough{\lstinline!TDouble!} the macro
  is \passthrough{\lstinline!T\_DOUBLE!}. It will be expanded to a
  function which casts the argument to the pointer to the object. For
  example, \passthrough{\lstinline!T\_DOUBLE (obj)!} casts the type of
  \passthrough{\lstinline!obj!} to \passthrough{\lstinline!TDouble *!}.
\item
  \passthrough{\lstinline!<NAME SPACE>\_IS\_<NAME>!} macro is defined.
  For example, if the object is \passthrough{\lstinline!TDouble!} the
  macro is \passthrough{\lstinline!T\_IS\_DOUBLE!}. It will be expanded
  to a function which checks if the argument points the instance of
  \passthrough{\lstinline!TDouble!}. It returns true if the argument
  points a descendant of \passthrough{\lstinline!TDouble!}.
\item
  The class structure is defined. A final type object doesn't need to
  have its own member of class structure. The definition is like the
  line 11 to 14 in the \passthrough{\lstinline!example4.c!}.
\end{itemize}

You need to write the macro definition of the type of the object before
\passthrough{\lstinline!G\_DECLARE\_FINAL\_TYPE!}. For example, if the
object is \passthrough{\lstinline!TDouble!}, then

\begin{lstlisting}[language=C]
#define T_TYPE_DOUBLE  (t_double_get_type ())
\end{lstlisting}

needs to be defined before
\passthrough{\lstinline!G\_DECLARE\_FINAL\_TYPE!}.

The C file \passthrough{\lstinline!example5.c!} uses this macro. It
works like \passthrough{\lstinline!example3.c!} or
\passthrough{\lstinline!example4.c!}.

\begin{lstlisting}[language=C, numbers=left]
#include <glib-object.h>

#define T_TYPE_DOUBLE  (t_double_get_type ())
G_DECLARE_FINAL_TYPE (TDouble, t_double, T, DOUBLE, GObject)

struct _TDouble {
  GObject parent;
  double value;
};

G_DEFINE_TYPE (TDouble, t_double, G_TYPE_OBJECT)

static void
t_double_class_init (TDoubleClass *class) {
}

static void
t_double_init (TDouble *self) {
}

int
main (int argc, char **argv) {
  GType dtype;
  TDouble *d;

  dtype = t_double_get_type (); /* or dtype = T_TYPE_DOUBLE */
  if (dtype)
    g_print ("Registration was a success. The type is %lx.\n", dtype);
  else
    g_print ("Registration failed.\n");

  d = g_object_new (T_TYPE_DOUBLE, NULL);
  if (d)
    g_print ("Instantiation was a success. The instance address is %p.\n", d);
  else
    g_print ("Instantiation failed.\n");

  if (T_IS_DOUBLE (d))
    g_print ("d is TDouble instance.\n");
  else
    g_print ("d is not TDouble instance.\n");

  if (G_IS_OBJECT (d))
    g_print ("d is GObject instance.\n");
  else
    g_print ("d is not GObject instance.\n");
  g_object_unref (d);

  return 0;
}
\end{lstlisting}

Execute it.

\begin{lstlisting}
$ cd src/misc; _build/example5
Registration was a success. The type is 5560b4cf58a0.
Instantiation was a success. The instance address is 0x5560b4cf6400.
d is TDouble instance.
d is GObject instance.
\end{lstlisting}

\subsection{Separate the file into main.c, tdouble.h and
tdouble.c}\label{separate-the-file-into-main.c-tdouble.h-and-tdouble.c}

Now it's time to separate the contents into three files,
\passthrough{\lstinline!main.c!}, \passthrough{\lstinline!tdouble.h!}
and \passthrough{\lstinline!tdouble.c!}. An object is defined by two
files, a header file and C source file.

tdouble.h

\begin{lstlisting}[language=C, numbers=left]
#pragma once

#include <glib-object.h>

#define T_TYPE_DOUBLE  (t_double_get_type ())
G_DECLARE_FINAL_TYPE (TDouble, t_double, T, DOUBLE, GObject)

gboolean
t_double_get_value (TDouble *self, double *value);

void
t_double_set_value (TDouble *self, double value);

TDouble *
t_double_new (double value);
\end{lstlisting}

\begin{itemize}
\tightlist
\item
  Header files are public, i.e.~it is open to any files. Header files
  include macros, which gives type information, cast and type check, and
  public functions.
\item
  1: The directive \passthrough{\lstinline!\#pragma once!} prevent the
  compiler from reading the header file two times or more. It is not
  officially defined but is supported widely in many compilers.
\item
  5-6: \passthrough{\lstinline!T\_TYPE\_DOUBLE!} is public.
  \passthrough{\lstinline!G\_DECLARE\_FINAL\_TYPE!} is expanded to
  public definitions.
\item
  8-12: Public function declarations. They are getter and setter of the
  value of the object. They are called ``instance methods'', which are
  used in object-oriented languages.
\item
  14-15: Object instantiation function.
\end{itemize}

tdouble.c

\begin{lstlisting}[language=C, numbers=left]
#include "tdouble.h"

struct _TDouble {
  GObject parent;
  double value;
};

G_DEFINE_TYPE (TDouble, t_double, G_TYPE_OBJECT)

static void
t_double_class_init (TDoubleClass *class) {
}

static void
t_double_init (TDouble *self) {
}

gboolean
t_double_get_value (TDouble *self, double *value) {
  g_return_val_if_fail (T_IS_DOUBLE (self), FALSE);

  *value = self->value;
  return TRUE;
}

void
t_double_set_value (TDouble *self, double value) {
  g_return_if_fail (T_IS_DOUBLE (self));

  self->value = value;
}

TDouble *
t_double_new (double value) {
  TDouble *d;

  d = g_object_new (T_TYPE_DOUBLE, NULL);
  d->value = value;
  return d;
}
\end{lstlisting}

\begin{itemize}
\tightlist
\item
  3-6: Declaration of the instance structure. Since
  \passthrough{\lstinline!G\_DECLARE\_FINAL\_TYPE!} macro emits
  \passthrough{\lstinline!typedef struct \_TDouble TDouble!}, the tag
  name of the structure must be \passthrough{\lstinline!\_TDouble!}.
\item
  8: \passthrough{\lstinline!G\_DEFINE\_TYPE!} macro.
\item
  10-16: class and instance initialization functions. At present, they
  don't do anything.
\item
  18-24: Getter. The argument \passthrough{\lstinline!value!} is the
  pointer to a double type variable. Assigns the object value
  (\passthrough{\lstinline!self->value!}) to the variable. If it
  succeeds, it returns TRUE. The function
  \passthrough{\lstinline!g\_return\_val\_if\_fail!} is used to check
  the argument type. If the argument \passthrough{\lstinline!self!} is
  not TDouble type, it outputs error to the log and immediately returns
  FALSE. This function is used to report a programmer's error. You
  shouldn't use it for a runtime error. See
  \href{https://docs.gtk.org/glib/error-reporting.html}{Glib API
  Reference -- Error Reporting} for further information. The function
  \passthrough{\lstinline!g\_return\_val\_if\_fail!} isn't used in
  static class functions, which are private, because static functions
  are called only from functions in the same file and the caller knows
  the type of parameters.
\item
  26-31: Setter. The function
  \passthrough{\lstinline!g\_return\_if\_fail!} is used to check the
  argument type. This function doesn't return any value. Because the
  type of \passthrough{\lstinline!t\_double\_set\_value!} is
  \passthrough{\lstinline!void!} so no value will be returned.
  Therefore, we use \passthrough{\lstinline!g\_return\_if\_fail!}
  instead of \passthrough{\lstinline!g\_return\_val\_if\_fail!}.
\item
  33-40: Object instantiation function. It has one parameter
  \passthrough{\lstinline!value!} to set the value of the object.
\item
  37: This function uses \passthrough{\lstinline!g\_object\_new!} to
  instantiate the object. The argument
  \passthrough{\lstinline!T\_TYPE\_DOUBLE!} is expanded to a function
  \passthrough{\lstinline!t\_double\_get\_type ()!}. If this is the
  first call for \passthrough{\lstinline!t\_double\_get\_type!}, the
  type registration will be carried out.
\end{itemize}

main.c

\begin{lstlisting}[language=C, numbers=left]
#include <glib-object.h>
#include "tdouble.h"

int
main (int argc, char **argv) {
  TDouble *d;
  double value;

  d = t_double_new (10.0);
  if (t_double_get_value (d, &value))
    g_print ("t_double_get_value succesfully assigned %lf to value.\n", value);
  else
    g_print ("t_double_get_value failed.\n");

  t_double_set_value (d, -20.0);
  g_print ("Now, set d (tDouble object) with %lf.\n", -20.0);
  if (t_double_get_value (d, &value))
    g_print ("t_double_get_value succesfully assigned %lf to value.\n", value);
  else
    g_print ("t_double_get_value failed.\n");
  g_object_unref (d);

  return 0;
}
\end{lstlisting}

\begin{itemize}
\tightlist
\item
  2: Includes \passthrough{\lstinline!tdouble.h!}. This is necessary for
  accessing TDouble object.
\item
  9: Instantiate TDouble object and set \passthrough{\lstinline!d!} to
  point the object.
\item
  10-13: Tests the getter of the object.
\item
  15-20: Tests the setter of the object.
\item
  21: Releases the instance \passthrough{\lstinline!d!}.
\end{itemize}

The source files are located in src/tdouble1. Change your current
directory to the directory above and type the following.

\begin{lstlisting}
$ cd src/tdouble1
$ meson setup _build
$ ninja -C _build
\end{lstlisting}

Then, execute the program.

\begin{lstlisting}
$ cd src/tdouble1; _build/example6
t_double_get_value succesfully assigned 10.000000 to value.
Now, set d (tDouble object) with -20.000000.
t_double_get_value succesfully assigned -20.000000 to value.
\end{lstlisting}

This example is very simple. But any object has header file and C source
file like this. And they follow the convention. You probably aware of
the importance of the convention. For the further information refer to
\href{https://docs.gtk.org/gobject/concepts.html\#conventions}{GObject
API Reference -- Conventions}.

\subsection{Functions}\label{functions}

Functions of objects are open to other objects. They are like public
methods in object oriented languages. They are actually called
``instance method'' in the GObject API Reference.

It is natural to add calculation operators to TDouble objects because
they represent real numbers. For example,
\passthrough{\lstinline!t\_double\_add!} adds the value of the instance
and another instance. Then it creates a new TDouble instance which has a
value of the sum of them.

\begin{lstlisting}[language=C]
TDouble *
t_double_add (TDouble *self, TDouble *other) {
  g_return_val_if_fail (T_IS_DOUBLE (self), NULL);
  g_return_val_if_fail (T_IS_DOUBLE (other), NULL);
  double value;

  if (! t_double_get_value (other, &value))
    return NULL;
  return t_double_new (self->value + value);
}
\end{lstlisting}

The first argument \passthrough{\lstinline!self!} is the instance the
function belongs to. The second argument \passthrough{\lstinline!other!}
is another TDouble instance.

The value of \passthrough{\lstinline!self!} can be accessed by
\passthrough{\lstinline!self->value!}, but don't use
\passthrough{\lstinline!other->value!} to get the value of
\passthrough{\lstinline!other!}. Use a function
\passthrough{\lstinline!t\_double\_get\_value!} instead. Because
\passthrough{\lstinline!self!} is an instance out of
\passthrough{\lstinline!other!}. Generally, the structure of an object
isn't open to other objects. When an object A access to another object
B, A must use a public function provided by B.

\subsection{Exercise}\label{exercise}

Write functions of TDouble object for subtraction, multiplication,
division and sign changing (unary minus). Compare your program to
\passthrough{\lstinline!tdouble.c!} in src/tdouble2 directory.
