The GitHub page of this tutorial is available. Click
\href{https://toshiocp.github.io/Gobject-tutorial/}{here}.

\paragraph{About this tutorial}\label{about-this-tutorial}

GObject is the base system for the GTK library, the current version of
which is four. GTK provides GUI on Linux and is used by GNOME desktop
system and many applications. See
\href{https://github.com/ToshioCP/Gtk4-tutorial}{GTK 4 tutorial}. One of
the problem to understand GTK 4 is the difficulty of the GObject. This
tutorial is useful for those who learns GTK 4. And the readers of this
tutorial should read GTK 4 tutorial because GTK is the only application
of GObject so far.

\href{https://docs.gtk.org/gobject/}{GObject API Reference} offers
everything necessary for GObject. The contents of this tutorial are not
beyond the documentation. It just shows examples and how to write
GObject programs. But I believe it is useful for the beginners who feels
difficulty to learn the GObject system. Readers should refer to the
GObject documentation when learning this tutorial.

\paragraph{Generating GFM, HTML and
PDF}\label{generating-gfm-html-and-pdf}

The table of contents are at the end of this file and you can see all
the tutorials through the link. However, you can make GFM, HTML or PDF
by the following steps. GFM is `GitHub Flavored Markdown', which is used
in the document files in the GitHub repository.

\begin{enumerate}
\def\labelenumi{\arabic{enumi}.}
\tightlist
\item
  You need Linux operating system, ruby, rake, pandoc and LaTeX system.
\item
  download the
  \href{https://github.com/ToshioCP/Gobject-tutorial}{GObject-tutorial
  repository} and uncompress the files.
\item
  change your current directory to the top directory of the files.
\item
  type \passthrough{\lstinline!rake!} to produce GFM files. The files
  are generated under \passthrough{\lstinline!gfm!} directory.
\item
  type \passthrough{\lstinline!rake html!} to produce HTML files. The
  files are generated under \passthrough{\lstinline!docs!} directory.
\item
  type \passthrough{\lstinline!rake pdf!} to produce a PDF file. The
  file is generated under \passthrough{\lstinline!latex!} directory.
\end{enumerate}

This system is the same as the one in the
\passthrough{\lstinline!GTK 4 tutorial!} repository. There's a document
\passthrough{\lstinline!Readme\_for\_developers.md!} in
\passthrough{\lstinline!gfm!} directory in it. It describes the details.

\paragraph{Contribution}\label{contribution}

If you have any questions, feel free to post an issue. If you find any
mistakes in the tutorial, post an issue or pull-request. When you give a
pull-request, correct the source files, which are under the `src'
directory, and run \passthrough{\lstinline!rake!} and
\passthrough{\lstinline!rake html!}. Then GFM and HTML files are
automatically updated.
