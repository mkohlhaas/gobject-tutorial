\section{Interface}\label{interface}

Interface is similar to abstract class. Interface defines virtual
functions which are expected to be overridden by a function in another
instantiable object.

This section provides a simple example, TComparable. TComparable is an
interface. It defines functions to compare. They are:

\begin{itemize}
\tightlist
\item
  \passthrough{\lstinline!t\_comparable\_cmp (self, other)!}: It
  compares \passthrough{\lstinline!self!} and
  \passthrough{\lstinline!other!}. The first argument
  \passthrough{\lstinline!self!} is an instance on which
  \passthrough{\lstinline!t\_comparable\_cmp!} runs. The second argument
  \passthrough{\lstinline!other!} is another instance. This function
  needs to be overridden in an object which implements the interface.

  \begin{itemize}
  \tightlist
  \item
    If \passthrough{\lstinline!self!} is equal to
    \passthrough{\lstinline!other!},
    \passthrough{\lstinline!t\_comparable\_cmp!} returns 0.
  \item
    If \passthrough{\lstinline!self!} is greater than
    \passthrough{\lstinline!other!},
    \passthrough{\lstinline!t\_comparable\_cmp!} returns 1.
  \item
    If \passthrough{\lstinline!self!} is less than
    \passthrough{\lstinline!other!},
    \passthrough{\lstinline!t\_comparable\_cmp!} returns -1.
  \item
    If an error happens, \passthrough{\lstinline!t\_comparable\_cmp!}
    returns -2.
  \end{itemize}
\item
  \passthrough{\lstinline!t\_comparable\_eq (self, other)!}: It returns
  TRUE if \passthrough{\lstinline!self!} is equal to
  \passthrough{\lstinline!other!}. Otherwise it returns FALSE. You need
  to be careful that FALSE is returned even if an error occurs.
\item
  \passthrough{\lstinline!t\_comparable\_gt (self, other)!}: It returns
  TRUE if \passthrough{\lstinline!self!} is greater than
  \passthrough{\lstinline!other!}. Otherwise it returns FALSE.
\item
  \passthrough{\lstinline!t\_comparable\_lt (self, other)!}: It returns
  TRUE if \passthrough{\lstinline!self!} is less than
  \passthrough{\lstinline!other!}. Otherwise it returns FALSE.
\item
  \passthrough{\lstinline!t\_comparable\_ge (self, other)!}: It returns
  TRUE if \passthrough{\lstinline!self!} is greater than or equal to
  \passthrough{\lstinline!other!}. Otherwise it returns FALSE.
\item
  \passthrough{\lstinline!t\_comparable\_le (self, other)!}: It returns
  TRUE if \passthrough{\lstinline!self!} is less than or equal to
  \passthrough{\lstinline!other!}. Otherwise it returns FALSE.
\end{itemize}

Numbers and strings are comparable. TInt, TDouble and TStr implement
TComparable interface so that they can use the functions above. In
addition, TNumStr can use the functions because it is a child class of
TStr.

For example,

\begin{lstlisting}[language=C]
TInt *i1 = t_int_new_with_value (10);
TInt *i2 = t_int_new_with_value (20);
t_comparable_eq (T_COMPARABLE (i1), T_COMPARABLE (i2)); /* => FALSE */
t_comparable_lt (T_COMPARABLE (i1), T_COMPARABLE (i2)); /* => TRUE */
\end{lstlisting}

What's the difference between interface and abstract class? Virtual
functions in an abstract class are overridden by a function in its
descendant classes. Virtual functions in an interface are overridden by
a function in any classes. Compare TNumber and TComparable.

\begin{itemize}
\tightlist
\item
  A function \passthrough{\lstinline!t\_number\_add!} is overridden in
  TIntClass and TDoubleClass. It can't be overridden in TStrClass
  because TStr isn't a descendant of TNumber.
\item
  A function \passthrough{\lstinline!t\_comparable\_cmp!} is overridden
  in TIntClass, TDoubleClass and TStrClass.
\end{itemize}

\subsection{TComparable interface}\label{tcomparable-interface}

Defining interfaces is similar to defining objects.

\begin{itemize}
\tightlist
\item
  Use \passthrough{\lstinline!G\_DECLARE\_INTERFACE!} instead of
  \passthrough{\lstinline!G\_DECLARE\_FINAL\_TYPE!}.
\item
  Use \passthrough{\lstinline!G\_DEFINE\_INTERFACE!} instead of
  \passthrough{\lstinline!G\_DEFINE\_TYPE!}.
\end{itemize}

Now let's see the header file.

\begin{lstlisting}[language=C, numbers=left]
#pragma once

#include <glib-object.h>

#define T_TYPE_COMPARABLE  (t_comparable_get_type ())
G_DECLARE_INTERFACE (TComparable, t_comparable, T, COMPARABLE, GObject)

struct _TComparableInterface {
  GTypeInterface parent;
  /* signal */
  void (*arg_error) (TComparable *self);
  /* virtual function */
  int (*cmp) (TComparable *self, TComparable *other);
};

/* t_comparable_cmp */
/* if self > other, then returns 1 */
/* if self = other, then returns 0 */
/* if self < other, then returns -1 */
/* if error happens, then returns -2 */

int
t_comparable_cmp (TComparable *self, TComparable *other);

gboolean
t_comparable_eq (TComparable *self, TComparable *other);

gboolean
t_comparable_gt (TComparable *self, TComparable *other);

gboolean
t_comparable_lt (TComparable *self, TComparable *other);

gboolean
t_comparable_ge (TComparable *self, TComparable *other);

gboolean
t_comparable_le (TComparable *self, TComparable *other);
\end{lstlisting}

\begin{itemize}
\tightlist
\item
  6: \passthrough{\lstinline!G\_DECLARE\_INTERFACE!} macro. The last
  parameter is a prerequisite of the interface. The prerequisite of
  TComparable is GObject. So, any other object than the descendants of
  GObject, for example GVariant, can't implement TComparable. A
  prerequisite is the GType of either an interface or a class. This
  macro expands to:

  \begin{itemize}
  \tightlist
  \item
    Declaration of \passthrough{\lstinline!t\_comparable\_get\_type()!}.
  \item
    \passthrough{\lstinline!Typedef struct \_TComparableInterface TComparableInterface!}
  \item
    \passthrough{\lstinline!T\_COMPARABLE ()!} macro. It casts an
    instance to TComparable type.
  \item
    \passthrough{\lstinline!T\_IS\_COMPARABLE ()!} macro. It checks if
    the type of an instance is
    \passthrough{\lstinline!T\_TYPE\_COMPARABLE!}.
  \item
    \passthrough{\lstinline!T\_COMPARABLE\_GET\_IFACE ()!} macro. It
    gets the interface of the instance which is given as an argument.
  \end{itemize}
\item
  8-14: \passthrough{\lstinline!TComparableInterface!} structure. This
  is similar to a class structure. The first member is the parent
  interface. The parent of
  \passthrough{\lstinline!TComparableInterface!} is
  \passthrough{\lstinline!GTypeInterface!}.
  \passthrough{\lstinline!GTypeInterface!} is a base of all interface
  types. It is like a \passthrough{\lstinline!GTypeClass!} which is a
  base of all class types. \passthrough{\lstinline!GTypeClass!} is the
  first member of the structure \passthrough{\lstinline!GObjectClass!}.
  (See \passthrough{\lstinline!gobject.h!}. Note that
  \passthrough{\lstinline!GObjectClass!} is the same as
  \passthrough{\lstinline!struct \_GObjectClass!}.) The next member is a
  pointer \passthrough{\lstinline!arg\_error!} to the default signal
  handler of ``arg-error'' signal. This signal is emitted when the
  second argument of the comparison function is inappropriate. For
  example, if \passthrough{\lstinline!self!} is TInt and
  \passthrough{\lstinline!other!} is TStr, both of them are Comparable
  instance. But they are \emph{not} able to compare. This is because
  \passthrough{\lstinline!other!} isn't TNumber. The last member
  \passthrough{\lstinline!cmp!} is a pointer to a comparison method. It
  is a virtual function and is expected to be overridden by a function
  in the object which implements the interface.
\item
  22-38: Public functions.
\end{itemize}

C file \passthrough{\lstinline!tcomparable.c!} is as follows.

\begin{lstlisting}[language=C, numbers=left]
#include "tcomparable.h"

static guint t_comparable_signal;

G_DEFINE_INTERFACE (TComparable, t_comparable, G_TYPE_OBJECT)

static void
arg_error_default_cb (TComparable *self) {
  g_printerr ("\nTComparable: argument error.\n");
}

static void
t_comparable_default_init (TComparableInterface *iface) {
  /* virtual function */
  iface->cmp = NULL;
  /* argument error signal */
  iface->arg_error = arg_error_default_cb;
  t_comparable_signal =
  g_signal_new ("arg-error",
                T_TYPE_COMPARABLE,
                G_SIGNAL_RUN_LAST | G_SIGNAL_NO_RECURSE | G_SIGNAL_NO_HOOKS,
                G_STRUCT_OFFSET (TComparableInterface, arg_error),
                NULL /* accumulator */,
                NULL /* accumulator data */,
                NULL /* C marshaller */,
                G_TYPE_NONE /* return_type */,
                0     /* n_params */
                );
}

int
t_comparable_cmp (TComparable *self, TComparable *other) {
  g_return_val_if_fail (T_IS_COMPARABLE (self), -2);

  TComparableInterface *iface = T_COMPARABLE_GET_IFACE (self);
  
  return (iface->cmp == NULL ? -2 : iface->cmp (self, other));
}

gboolean
t_comparable_eq (TComparable *self, TComparable *other) {
  return (t_comparable_cmp (self, other) == 0);
}

gboolean
t_comparable_gt (TComparable *self, TComparable *other) {
  return (t_comparable_cmp (self, other) == 1);
}

gboolean
t_comparable_lt (TComparable *self, TComparable *other) {
  return (t_comparable_cmp (self, other) == -1);
}

gboolean
t_comparable_ge (TComparable *self, TComparable *other) {
  int result = t_comparable_cmp (self, other);
  return (result == 1 || result == 0);
}

gboolean
t_comparable_le (TComparable *self, TComparable *other) {
  int result = t_comparable_cmp (self, other);
  return (result == -1 || result == 0);
}
\end{lstlisting}

\begin{itemize}
\tightlist
\item
  5: \passthrough{\lstinline!G\_DEFINE\_INTERFACE!} macro. The third
  parameter is the type of the prerequisite. This macro expands to:

  \begin{itemize}
  \tightlist
  \item
    Declaration of
    \passthrough{\lstinline!t\_comparable\_default\_init ()!}.
  \item
    Definition of \passthrough{\lstinline!t\_comparable\_get\_type ()!}.
  \end{itemize}
\item
  7-10: \passthrough{\lstinline!arg\_error\_default\_cb!} is a default
  signal handler of ``arg-error'' signal.
\item
  12- 29: \passthrough{\lstinline!t\_comparable\_default\_init!}
  function. This function is similar to class initialization function.
  It initializes \passthrough{\lstinline!TComparableInterface!}
  structure.
\item
  15: Assigns NULL to \passthrough{\lstinline!cmp!}. So, the comparison
  method doesn't work before an implementation class overrides it.
\item
  17: Set the default signal handler of the signal ``arg-error''.
\item
  18-28: Creates a signal ``arg-error''.
\item
  31-38: The function \passthrough{\lstinline!t\_comparable\_cmp!}. It
  checks the type of \passthrough{\lstinline!self!} on the first line.
  If it isn't comparable, it logs the error message and returns -2
  (error). If \passthrough{\lstinline!iface->cmp!} is NULL (it means the
  class method hasn't been overridden), then it returns -2. Otherwise it
  calls the class method and returns the value returned by the class
  method.
\item
  40-65: Public functions. These five functions are based on
  \passthrough{\lstinline!t\_comparable\_cmp!}. Therefore, no overriding
  is necessary for them. For example,
  \passthrough{\lstinline!t\_comparable\_eq!} just calls
  \passthrough{\lstinline!t\_comparable\_cmp!}. And it returns TRUE if
  \passthrough{\lstinline!t\_comparable\_cmp!} returns zero. Otherwise
  it returns FALSE.
\end{itemize}

This program uses a signal to give the argument type error information
to a user. This error is usually a program error rather than a run-time
error. Using a signal to report a program error is not a good way. The
best way is using \passthrough{\lstinline!g\_return\_if\_fail!}. The
reason why I made this signal is just I wanted to show how to implement
a signal in interfaces.

\subsection{Implementing interface}\label{implementing-interface}

TInt, TDouble and TStr implement TComparable. First, look at TInt. The
header file is the same as before. The implementation is written in C
file.

\passthrough{\lstinline!tint.c!} is as follows.

\begin{lstlisting}[language=C, numbers=left]
#include "../tnumber/tnumber.h"
#include "../tnumber/tint.h"
#include "../tnumber/tdouble.h"
#include "tcomparable.h"

enum {
  PROP_0,
  PROP_INT,
  N_PROPERTIES
};

static GParamSpec *int_properties[N_PROPERTIES] = {NULL, };

struct _TInt {
  TNumber parent;
  int value;
};

static void t_comparable_interface_init (TComparableInterface *iface);

G_DEFINE_TYPE_WITH_CODE (TInt, t_int, T_TYPE_NUMBER,
                         G_IMPLEMENT_INTERFACE (T_TYPE_COMPARABLE, t_comparable_interface_init))

static int
t_int_comparable_cmp (TComparable *self, TComparable *other) {
  if (! T_IS_NUMBER (other)) {
    g_signal_emit_by_name (self, "arg-error");
    return -2;
  }

  int i;
  double s, o;

  s = (double) T_INT (self)->value;
  if (T_IS_INT (other)) {
    g_object_get (other, "value", &i, NULL);
    o = (double) i;
  } else
    g_object_get (other, "value", &o, NULL);
  if (s > o)
    return 1;
  else if (s == o)
    return 0;
  else if (s < o)
    return -1;
  else /* This can't happen. */
    return -2;
}

static void
t_comparable_interface_init (TComparableInterface *iface) {
  iface->cmp = t_int_comparable_cmp;
}

static void
t_int_set_property (GObject *object, guint property_id, const GValue *value, GParamSpec *pspec) {
  TInt *self = T_INT (object);

  if (property_id == PROP_INT)
    self->value = g_value_get_int (value);
  else
    G_OBJECT_WARN_INVALID_PROPERTY_ID (object, property_id, pspec);
}

static void
t_int_get_property (GObject *object, guint property_id, GValue *value, GParamSpec *pspec) {
  TInt *self = T_INT (object);

  if (property_id == PROP_INT)
    g_value_set_int (value, self->value);
  else
    G_OBJECT_WARN_INVALID_PROPERTY_ID (object, property_id, pspec);
}

static void
t_int_init (TInt *self) {
}

/* arithmetic operator */
/* These operators create a new instance and return a pointer to it. */
#define t_int_binary_op(op) \
  int i; \
  double d; \
  if (T_IS_INT (other)) { \
    g_object_get (T_INT (other), "value", &i, NULL); \
    return  T_NUMBER (t_int_new_with_value (T_INT(self)->value op i)); \
  } else { \
    g_object_get (T_DOUBLE (other), "value", &d, NULL); \
    return  T_NUMBER (t_int_new_with_value (T_INT(self)->value op (int) d)); \
  }

static TNumber *
t_int_add (TNumber *self, TNumber *other) {
  g_return_val_if_fail (T_IS_INT (self), NULL);

  t_int_binary_op (+)
}

static TNumber *
t_int_sub (TNumber *self, TNumber *other) {
  g_return_val_if_fail (T_IS_INT (self), NULL);

  t_int_binary_op (-)
}

static TNumber *
t_int_mul (TNumber *self, TNumber *other) {
  g_return_val_if_fail (T_IS_INT (self), NULL);

  t_int_binary_op (*)
}

static TNumber *
t_int_div (TNumber *self, TNumber *other) {
  g_return_val_if_fail (T_IS_INT (self), NULL);

  int i;
  double d;

  if (T_IS_INT (other)) {
    g_object_get (T_INT (other), "value", &i, NULL);
    if (i == 0) {
      g_signal_emit_by_name (self, "div-by-zero");
      return NULL;
    } else
      return  T_NUMBER (t_int_new_with_value (T_INT(self)->value / i));
  } else {
    g_object_get (T_DOUBLE (other), "value", &d, NULL);
    if (d == 0) {
      g_signal_emit_by_name (self, "div-by-zero");
      return NULL;
    } else
      return  T_NUMBER (t_int_new_with_value (T_INT(self)->value / (int)  d));
  }
}

static TNumber *
t_int_uminus (TNumber *self) {
  g_return_val_if_fail (T_IS_INT (self), NULL);

  return T_NUMBER (t_int_new_with_value (- T_INT(self)->value));
}

static char *
t_int_to_s (TNumber *self) {
  g_return_val_if_fail (T_IS_INT (self), NULL);

  int i;

  g_object_get (T_INT (self), "value", &i, NULL); 
  return g_strdup_printf ("%d", i);
}

static void
t_int_class_init (TIntClass *class) {
  TNumberClass *tnumber_class = T_NUMBER_CLASS (class);
  GObjectClass *gobject_class = G_OBJECT_CLASS (class);

  /* override virtual functions */
  tnumber_class->add = t_int_add;
  tnumber_class->sub = t_int_sub;
  tnumber_class->mul = t_int_mul;
  tnumber_class->div = t_int_div;
  tnumber_class->uminus = t_int_uminus;
  tnumber_class->to_s = t_int_to_s;

  gobject_class->set_property = t_int_set_property;
  gobject_class->get_property = t_int_get_property;
  int_properties[PROP_INT] = g_param_spec_int ("value", "val", "Integer value", G_MININT, G_MAXINT, 0, G_PARAM_READWRITE);
  g_object_class_install_properties (gobject_class, N_PROPERTIES, int_properties);
}

TInt *
t_int_new_with_value (int value) {
  TInt *i;

  i = g_object_new (T_TYPE_INT, "value", value, NULL);
  return i;
}

TInt *
t_int_new (void) {
  TInt *i;

  i = g_object_new (T_TYPE_INT, NULL);
  return i;
}
\end{lstlisting}

\begin{itemize}
\tightlist
\item
  4: It needs to include the header file of TComparable.
\item
  19: Declaration of
  \passthrough{\lstinline!t\_comparable\_interface\_init ()!} function.
  This declaration must be done before
  \passthrough{\lstinline!G\_DEFINE\_TYPE\_WITH\_CODE!} macro.
\item
  21-22: \passthrough{\lstinline!G\_DEFINE\_TYPE\_WITH\_CODE!} macro.
  The last parameter is
  \passthrough{\lstinline!G\_IMPLEMENT\_INTERFACE!} macro. The second
  parameter of \passthrough{\lstinline!G\_IMPLEMENT\_INTERFACE!} is
  \passthrough{\lstinline!t\_comparable\_interface\_init!}. These two
  macros expands to:

  \begin{itemize}
  \tightlist
  \item
    Declaration of \passthrough{\lstinline!t\_int\_class\_init ()!}.
  \item
    Declaration of \passthrough{\lstinline!t\_int\_init ()!}.
  \item
    Definition of \passthrough{\lstinline!t\_int\_parent\_class!} static
    variable which points to the parent's class.
  \item
    Definition of \passthrough{\lstinline!t\_int\_get\_type ()!}. This
    function includes
    \passthrough{\lstinline!g\_type\_register\_static\_simple ()!} and
    \passthrough{\lstinline!g\_type\_add\_interface\_static ()!}. The
    function
    \passthrough{\lstinline!g\_type\_register\_static\_simple ()!} is a
    convenient version of
    \passthrough{\lstinline!g\_type\_register\_static ()!}. It registers
    TInt type to the type system. The function
    \passthrough{\lstinline!g\_type\_add\_interface\_static ()!} adds an
    interface type to an instance type. There is a good example in
    \href{https://docs.gtk.org/gobject/concepts.html\#non-instantiatable-classed-types-interfaces}{GObject
    Reference Manual, Interfaces}. If you want to know how to write
    codes without the macros, see
    \passthrough{\lstinline!tint\_without\_macro.c!}.
  \end{itemize}
\item
  24-48: \passthrough{\lstinline!t\_int\_comparable\_cmp!} is a function
  to compare TInt instance to TNumber instance.
\item
  26-29: Checks the type of \passthrough{\lstinline!other!}. If the
  argument type is not TNumber, it emits ``arg-error'' signal with
  \passthrough{\lstinline!g\_signal\_emit\_by\_name!}.
\item
  34: Converts \passthrough{\lstinline!self!} into double.
\item
  35-39: Gets the value of \passthrough{\lstinline!other!} and if it is
  TInt then the value is casted to double.
\item
  40-47: compares \passthrough{\lstinline!s!} and
  \passthrough{\lstinline!o!} and returns 1, 0, -1 and -2.
\item
  50-53: \passthrough{\lstinline!t\_comparable\_interface\_init!}. This
  function is called in the initialization process of TInt. The function
  \passthrough{\lstinline!t\_int\_comparable\_cmp!} is assigned to
  \passthrough{\lstinline!iface->cmp!}.
\end{itemize}

\passthrough{\lstinline!tdouble.c!} is almost same as
\passthrough{\lstinline!tint.c!}. These two objects can be compared
because int is casted to double before the comparison.

\passthrough{\lstinline!tstr.c!} is as follows.

\begin{lstlisting}[language=C, numbers=left]
#include "../tstr/tstr.h"
#include "tcomparable.h"

enum {
  PROP_0,
  PROP_STRING,
  N_PROPERTIES
};

static GParamSpec *str_properties[N_PROPERTIES] = {NULL, };

typedef struct {
  char *string;
} TStrPrivate;

static void t_comparable_interface_init (TComparableInterface *iface);

G_DEFINE_TYPE_WITH_CODE (TStr, t_str, G_TYPE_OBJECT,
                         G_ADD_PRIVATE (TStr)
                         G_IMPLEMENT_INTERFACE (T_TYPE_COMPARABLE, t_comparable_interface_init))

static void
t_str_set_property (GObject *object, guint property_id, const GValue *value, GParamSpec *pspec) {
  TStr *self = T_STR (object);

/* The returned value of the function g_value_get_string can be NULL. */
/* The function t_str_set_string calls a class method, */
/* which is expected to rewrite in the descendant object. */
  if (property_id == PROP_STRING)
    t_str_set_string (self, g_value_get_string (value));
  else
    G_OBJECT_WARN_INVALID_PROPERTY_ID (object, property_id, pspec);
}

static void
t_str_get_property (GObject *object, guint property_id, GValue *value, GParamSpec *pspec) {
  TStr *self = T_STR (object);
  TStrPrivate *priv = t_str_get_instance_private (self);

/* The second argument of the function g_value_set_string can be NULL. */
  if (property_id == PROP_STRING)
    g_value_set_string (value, priv->string);
  else
    G_OBJECT_WARN_INVALID_PROPERTY_ID (object, property_id, pspec);
}

/* This function just set the string. */
/* So, no notify signal is emitted. */
static void
t_str_real_set_string (TStr *self, const char *s) {
  TStrPrivate *priv = t_str_get_instance_private (self);

  if (priv->string)
    g_free (priv->string);
  priv->string = g_strdup (s);
}

static void
t_str_finalize (GObject *object) {
  TStr *self = T_STR (object);
  TStrPrivate *priv = t_str_get_instance_private (self);

  if (priv->string)
    g_free (priv->string);
  G_OBJECT_CLASS (t_str_parent_class)->finalize (object);
}

static int
t_str_comparable_cmp (TComparable *self, TComparable *other) {
  if (! T_IS_STR (other)) {
    g_signal_emit_by_name (self, "arg-error");
    return -2;
  }

  char *s, *o;
  int result;

  s = t_str_get_string (T_STR (self));
  o = t_str_get_string (T_STR (other));

  if (strcmp (s, o) > 0)
    result = 1;
  else if (strcmp (s, o) == 0)
    result = 0;
  else if (strcmp (s, o) < 0)
    result = -1;
  else /* This can't happen. */
    result = -2;
  g_free (s);
  g_free (o);
  return result;
}

static void
t_comparable_interface_init (TComparableInterface *iface) {
  iface->cmp = t_str_comparable_cmp;
}

static void
t_str_init (TStr *self) {
  TStrPrivate *priv = t_str_get_instance_private (self);

  priv->string = NULL;
}

static void
t_str_class_init (TStrClass *class) {
  GObjectClass *gobject_class = G_OBJECT_CLASS (class);

  gobject_class->finalize = t_str_finalize;
  gobject_class->set_property = t_str_set_property;
  gobject_class->get_property = t_str_get_property;
  str_properties[PROP_STRING] = g_param_spec_string ("string", "str", "string", "", G_PARAM_READWRITE);
  g_object_class_install_properties (gobject_class, N_PROPERTIES, str_properties);

  class->set_string = t_str_real_set_string;
}

/* setter and getter */
void
t_str_set_string (TStr *self, const char *s) {
  g_return_if_fail (T_IS_STR (self));
  TStrClass *class = T_STR_GET_CLASS (self);

/* The setter calls the class method 'set_string', */
/* which is expected to be overridden by the descendant TNumStr. */
/* Therefore, the behavior of the setter is different between TStr and TNumStr. */
  class->set_string (self, s);
}

char *
t_str_get_string (TStr *self) {
  g_return_val_if_fail (T_IS_STR (self), NULL);
  TStrPrivate *priv = t_str_get_instance_private (self);

  return g_strdup (priv->string);
}

TStr *
t_str_concat (TStr *self, TStr *other) {
  g_return_val_if_fail (T_IS_STR (self), NULL);
  g_return_val_if_fail (T_IS_STR (other), NULL);

  char *s1, *s2, *s3;
  TStr *str;

  s1 = t_str_get_string (self);
  s2 = t_str_get_string (other);
  if (s1 && s2)
    s3 = g_strconcat (s1, s2, NULL);
  else if (s1)
    s3 = g_strdup (s1);
  else if (s2)
    s3 = g_strdup (s2);
  else
    s3 = NULL;
  str = t_str_new_with_string (s3);
  if (s1) g_free (s1);
  if (s2) g_free (s2);
  if (s3) g_free (s3);
  return str;
}

/* create a new TStr instance */
TStr *
t_str_new_with_string (const char *s) {
  return T_STR (g_object_new (T_TYPE_STR, "string", s, NULL));
}

TStr *
t_str_new (void) {
  return T_STR (g_object_new (T_TYPE_STR, NULL));
}
\end{lstlisting}

\begin{itemize}
\tightlist
\item
  16: Declares \passthrough{\lstinline!t\_comparable\_interface\_init!}
  function. It needs to be declared before
  \passthrough{\lstinline!G\_DEFINE\_TYPE\_WITH\_CODE!} macro.
\item
  18-20: \passthrough{\lstinline!G\_DEFINE\_TYPE\_WITH\_CODE!} macro.
  Because TStr is derivable type, its private area (TStrPrivate) is
  needed. \passthrough{\lstinline!G\_ADD\_PRIVATE!} macro makes the
  private area. Be careful that there's no comma after
  \passthrough{\lstinline!G\_ADD\_PRIVATE!} macro.
\item
  68-92: \passthrough{\lstinline!t\_str\_comparable\_cmp!}.
\item
  70-73: Checks the type of \passthrough{\lstinline!other!}. If it is
  not TStr, ``arg-error'' signal is emitted.
\item
  78-79: Gets strings \passthrough{\lstinline!s!} and
  \passthrough{\lstinline!o!} from TStr objects
  \passthrough{\lstinline!self!} and \passthrough{\lstinline!other!}.
\item
  81-88: Compares \passthrough{\lstinline!s!} and
  \passthrough{\lstinline!o!} with the C standard function
  \passthrough{\lstinline!strcmp!}.
\item
  89-90: Frees \passthrough{\lstinline!s!} and
  \passthrough{\lstinline!o!}.
\item
  91: Returns the result.
\item
  94-97: \passthrough{\lstinline!t\_comparable\_interface\_init!}
  function. It overrides \passthrough{\lstinline!iface->comp!} with
  \passthrough{\lstinline!t\_str\_comparable\_cmp!}.
\end{itemize}

TStr can be compared with TStr, but not with TInt nor TDouble.
Generally, comparison is available between two same type instances.

TNumStr itself doesn't implement TComparable. But it is a child of TStr,
so it is comparable. The comparison is based on the alphabetical order.
So, ``a'' is bigger than ``b'' and ``three'' is bigger than ``two''.

\subsection{Test program.}\label{test-program.}

\passthrough{\lstinline!main.c!} is a test program.

\begin{lstlisting}[language=C, numbers=left]
#include <glib-object.h>
#include "tcomparable.h"
#include "../tnumber/tnumber.h"
#include "../tnumber/tint.h"
#include "../tnumber/tdouble.h"
#include "../tstr/tstr.h"

static void
t_print (const char *cmp, TComparable *c1, TComparable *c2) {
  char *s1, *s2;
  TStr *ts1, *ts2, *ts3;

  ts1 = t_str_new_with_string("\"");
  if (T_IS_NUMBER (c1))
    s1 = t_number_to_s (T_NUMBER (c1));
  else if (T_IS_STR (c1)) {
    ts2 = t_str_concat (ts1, T_STR (c1));
    ts3 = t_str_concat (ts2, ts1);
    s1 = t_str_get_string (T_STR (ts3));
    g_object_unref (ts2);
    g_object_unref (ts3);
  } else {
    g_print ("c1 isn't TInt, TDouble nor TStr.\n");
    return;
  }
  if (T_IS_NUMBER (c2))
    s2 = t_number_to_s (T_NUMBER (c2));
  else if (T_IS_STR (c2)) {
    ts2 = t_str_concat (ts1, T_STR (c2));
    ts3 = t_str_concat (ts2, ts1);
    s2 = t_str_get_string (T_STR (ts3));
    g_object_unref (ts2);
    g_object_unref (ts3);
  } else {
    g_print ("c2 isn't TInt, TDouble nor TStr.\n");
    return;
  }
  g_print ("%s %s %s.\n", s1, cmp, s2);
  g_object_unref (ts1);
  g_free (s1);
  g_free (s2);
}    

static void
compare (TComparable *c1, TComparable *c2) {
  if (t_comparable_eq (c1, c2))
    t_print ("equals", c1, c2);
  else if (t_comparable_gt (c1, c2))
    t_print ("is greater than", c1, c2);
  else if (t_comparable_lt (c1, c2))
    t_print ("is less than", c1, c2);
  else if (t_comparable_ge (c1, c2))
    t_print ("is greater than or equal to", c1, c2);
  else if (t_comparable_le (c1, c2))
    t_print ("is less than or equal to", c1, c2);
  else
    t_print ("can't compare to", c1, c2);
}

int
main (int argc, char **argv) {
  const char *one = "one";
  const char *two = "two";
  const char *three = "three";
  TInt *i;
  TDouble *d;
  TStr *str1, *str2, *str3;

  i = t_int_new_with_value (124);
  d = t_double_new_with_value (123.45);
  str1 = t_str_new_with_string (one);
  str2 = t_str_new_with_string (two);
  str3 = t_str_new_with_string (three);

  compare (T_COMPARABLE (i), T_COMPARABLE (d));
  compare (T_COMPARABLE (str1), T_COMPARABLE (str2));
  compare (T_COMPARABLE (str2), T_COMPARABLE (str3));
  compare (T_COMPARABLE (i), T_COMPARABLE (str1));

  g_object_unref (i);
  g_object_unref (d);
  g_object_unref (str1);
  g_object_unref (str2);
  g_object_unref (str3);

  return 0;
}
\end{lstlisting}

\begin{itemize}
\tightlist
\item
  8-42: The function \passthrough{\lstinline!t\_print!} has three
  parameters and builds an output string, then shows it to the display.
  When it builds the output, strings are surrounded with double quotes.
\item
  44-58: The function \passthrough{\lstinline!compare!} compares two
  TComparable objects and calls \passthrough{\lstinline!t\_print!} to
  display the result.
\item
  60-87: \passthrough{\lstinline!main!} function.
\item
  69-73: Creates TInt, TDouble and three TStr instances. They are given
  values.
\item
  75: Compares TInt and TDouble.
\item
  76-77: Compares two TStr.
\item
  78: Compare TInt to TStr. This makes an ``arg-error''.
\item
  80-84 Frees objects.
\end{itemize}

\subsection{Compilation and execution}\label{compilation-and-execution}

Change your current directory to src/tcomparable.

\begin{lstlisting}
$ cd src/tcomparable
$ meson setup _build
$ ninja -C _build
\end{lstlisting}

Then execute it.

\begin{lstlisting}
$ _build/tcomparable
124 is greater than 123.450000.
"one" is less than "two".
"two" is greater than "three".

TComparable: argument error.

TComparable: argument error.

TComparable: argument error.

TComparable: argument error.

TComparable: argument error.
124 can't compare to "one".
\end{lstlisting}

\subsection{Build an interface without
macros}\label{build-an-interface-without-macros}

We used macros such as \passthrough{\lstinline!G\_DECLARE\_INTERFACE!},
\passthrough{\lstinline!G\_DEFINE\_INTERFACE!} to build an interface.
And \passthrough{\lstinline!G\_DEFINE\_TYPE\_WITH\_CODE!} to implement
the interface. We can also build it without macros. There are three
files in the \passthrough{\lstinline!tcomparable!} directory.

\begin{itemize}
\tightlist
\item
  \passthrough{\lstinline!tcomparable\_without\_macro.h!}
\item
  \passthrough{\lstinline!tcomparable\_without\_macro.c!}
\item
  \passthrough{\lstinline!tint\_without\_macro.c!}
\end{itemize}

They don't use macros. Instead, they register the interface or
implementation of the interface to the type system directly. If you want
to know that, see the source files in src/tcomparable.

\subsection{GObject system and object oriented
languages}\label{gobject-system-and-object-oriented-languages}

If you know any object oriented languages, you probably have thought
that GObject and the languages are similar. Learning such languages is
very useful to know GObject system.
