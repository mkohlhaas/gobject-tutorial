\section{Prerequisites and License}\label{prerequisites-and-license}

\subsection{Prerequisites}\label{prerequisites}

\subsubsection{Tutorial document}\label{tutorial-document}

This tutorial is about GObject libraries. It is originally used on Linux
with C compiler, but now it is used more widely, on windows and macOS,
with Vala, python and so on. However, this tutorial describes only
\emph{C programs on Linux}.

If you want to try to compile the examples in the tutorial, you need:

\begin{itemize}
\tightlist
\item
  PC with Linux distribution like Ubuntu, Debian and so on.
\item
  Gcc
\item
  GLib. The version at the time this document is described is 2.76.1.
  Some example program needs version 2.74 or higher. But they work on
  the older version if you replace the new function or macro with the
  old one.
\item
  pkg-config
\item
  meson and ninja
\end{itemize}

Common Linux distributions has GLib, which is enough for you to compile
the examples in this repository.

\subsubsection{Tools to make GFM, HTML and PDF
files}\label{tools-to-make-gfm-html-and-pdf-files}

This repository includes ruby programs. They are used to create
Markdown(GFM) files, HTML files, LaTeX files and a PDF file.

You need:

\begin{itemize}
\tightlist
\item
  Linux distribution like Ubuntu.
\item
  Ruby programming language. There are two ways to install it. One is
  installing the distribution's package. The other is using rbenv and
  ruby-build. If you want to use the latest version of Ruby, you will
  need rbenv and ruby-build.
\item
  Rake. It is a gem, which is a library written in ruby. Ruby package
  includes Rake gem as a standard library so you don't need to install
  it separately.
\end{itemize}

\subsection{License}\label{license}

Copyright (C) 2021-2022 ToshioCP (Toshio Sekiya)

GObject tutorial repository contains the tutorial document and software
such as converters, generators and controllers. All of them make up the
`GObject tutorial' package. This package is simply called `GObject
tutorial' in the following description.

`GObject tutorial' is free; you can redistribute it and/or modify it
under the terms of the following licenses.

\begin{itemize}
\tightlist
\item
  The license of documents in GObject tutorial is the GNU Free
  Documentation License as published by the Free Software Foundation;
  either version 1.3 of the License or, at your opinion, any later
  version. The documents are Markdown, HTML and image files. If you
  generate a PDF file by running \passthrough{\lstinline!rake pdf!}, it
  is also included by the documents.
\item
  The license of programs in GObject tutorial is the GNU General Public
  License as published by the Free Software Foundation; either version 3
  of the License or, at your option, any later version. The programs are
  written in C, Ruby and other languages.
\end{itemize}

GObject tutorial is distributed in the hope that it will be useful, but
WITHOUT ANY WARRANTY; without even the implied warranty of
MERCHANTABILITY or FITNESS FOR A PARTICULAR PURPOSE. See the GNU License
web pages for more details.

\begin{itemize}
\tightlist
\item
  \href{https://www.gnu.org/licenses/fdl-1.3.html}{GNU Free
  Documentation License}
\item
  \href{https://www.gnu.org/licenses/gpl-3.0.html}{GNU General Public
  License}
\end{itemize}

The licenses above is effective since 15/August/2023. Before that, GPL
covered all the contents of the GObject tutorial. But GFDL1.3 is more
appropriate for documents so the license was changed. The license above
is the only effective license since 15/August/2023.
