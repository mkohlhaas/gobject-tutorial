\section{Signals}\label{signals}

\subsection{Signals}\label{signals-1}

Signals provide a means of communication between objects. Signals are
emitted when something happens or completes.

The steps to program a signal is shown below.

\begin{enumerate}
\def\labelenumi{\arabic{enumi}.}
\tightlist
\item
  Register a signal. A signal belongs to an object, so the registration
  is done in the class initialization function of the object.
\item
  Write a signal handler. A signal handler is a function that is invoked
  when the signal is emitted.
\item
  Connect the signal and handler. Signals are connected to handlers with
  \passthrough{\lstinline!g\_connect\_signal!} or its family functions.
\item
  Emit the signal.
\end{enumerate}

Step one and Four are done on the object to which the signal belongs.
Step three is usually done outside the object.

The process of signals is complicated and it takes long to explain all
the features. The contents of this section is limited to the minimum
things to write a simple signal and not necessarily accurate. If you
need an accurate information, refer to GObject API reference. There are
four parts which describe signals.

\begin{itemize}
\tightlist
\item
  \href{https://docs.gtk.org/gobject/concepts.html\#signals}{Type System
  Concepts -- signals}
\item
  \href{https://docs.gtk.org/gobject/\#functions}{Funcions
  (g\_signal\_XXX series)}
\item
  \href{https://docs.gtk.org/gobject/\#function_macros}{Funcions Macros
  (g\_signal\_XXX series)}
\item
  \href{https://docs.gtk.org/gobject/tutorial.html\#how-to-create-and-use-signals}{GObject
  Tutorial -- How to create and use signals}
\end{itemize}

\subsection{Signal registration}\label{signal-registration}

An example in this section is a signal emitted when division-by-zero
happens. First, we need to determine the name of the signal. Signal name
consists of letters, digits, dash (\passthrough{\lstinline!-!}) and
underscore (\passthrough{\lstinline!\_!}). The first character of the
name must be a letter. So, a string ``div-by-zero'' is appropriate for
the signal name.

There are four functions to register a signal. We will use
\href{https://docs.gtk.org/gobject/func.signal_new.html}{\passthrough{\lstinline!g\_signal\_new!}}
for ``div-by-zero'' signal.

\begin{lstlisting}[language=C]
guint
g_signal_new (const gchar *signal_name,
              GType itype,
              GSignalFlags signal_flags,
              guint class_offset,
              GSignalAccumulator accumulator,
              gpointer accu_data,
              GSignalCMarshaller c_marshaller,
              GType return_type,
              guint n_params,
              ...);
\end{lstlisting}

It needs a lot to explain each parameter. At present I just show you
\passthrough{\lstinline!g\_signal\_new!} function call extracted from
\passthrough{\lstinline!tdouble.c!}.

\begin{lstlisting}[language=C]
t_double_signal =
g_signal_new ("div-by-zero",
              G_TYPE_FROM_CLASS (class),
              G_SIGNAL_RUN_LAST | G_SIGNAL_NO_RECURSE | G_SIGNAL_NO_HOOKS,
              0 /* class offset.Subclass cannot override the class handler (default handler). */,
              NULL /* accumulator */,
              NULL /* accumulator data */,
              NULL /* C marshaller. g_cclosure_marshal_generic() will be used */,
              G_TYPE_NONE /* return_type */,
              0     /* n_params */
              );
\end{lstlisting}

\begin{itemize}
\tightlist
\item
  \passthrough{\lstinline!t\_double\_signal!} is a static guint
  variable. The type guint is the same as unsigned int. It is set to the
  signal id returned by the function
  \passthrough{\lstinline!g\_signal\_new!}.
\item
  The second parameter is the type (GType) of the object the signal
  belongs to. \passthrough{\lstinline!G\_TYPE\_FROM\_CLASS (class)!}
  returns the type corresponds to the class
  (\passthrough{\lstinline!class!} is a pointer to the class of the
  object).
\item
  The third parameter is a signal flag. Lots of pages are necessary to
  explain this flag. So, I want leave them out now. The argument above
  can be used in many cases. The definition is described in the
  \href{https://docs.gtk.org/gobject/flags.SignalFlags.html}{GObject API
  Reference -- SignalFlags}.
\item
  The return type is G\_TYPE\_NONE which means no value is returned by
  the signal handler.
\item
  \passthrough{\lstinline!n\_params!} is a number of parameters. This
  signal doesn't have parameters, so it is zero.
\end{itemize}

This function is located in the class initialization function
(\passthrough{\lstinline!t\_double\_class\_init!}).

You can use other functions such as
\passthrough{\lstinline!g\_signal\_newv!}. See
\href{https://docs.gtk.org/gobject/func.signal_newv.html}{GObject API
Reference} for details.

\subsection{Signal handler}\label{signal-handler}

Signal handler is a function that is called when the signal is emitted.
The handler has two parameters.

\begin{itemize}
\tightlist
\item
  The instance to which the signal belongs
\item
  A pointer to a user data which is given in the signal connection.
\end{itemize}

The ``div-by-zero'' signal doesn't need user data.

\begin{lstlisting}[language=C]
void div_by_zero_cb (TDouble *self, gpointer user_data) { ... ... ...}
\end{lstlisting}

The first argument \passthrough{\lstinline!self!} is the instance on
which the signal is emitted. You can leave out the second parameter.

\begin{lstlisting}[language=C]
void div_by_zero_cb (TDouble *self) { ... ... ...}
\end{lstlisting}

If a signal has parameters, the parameters are between the instance and
the user data. For example, the handler of ``window-added'' signal on
GtkApplication is:

\begin{lstlisting}[language=C]
void window_added (GtkApplication* self, GtkWindow* window, gpointer user_data);
\end{lstlisting}

The second argument \passthrough{\lstinline!window!} is the parameter of
the signal. The ``window-added'' signal is emitted when a new window is
added to the application. The parameter \passthrough{\lstinline!window!}
points a newly added window. See
\href{https://docs.gtk.org/gtk4/signal.Application.window-added.html}{GTK
API reference} for further information.

The handler of ``div-by-zero'' signal just shows an error message.

\begin{lstlisting}[language=C]
static void
div_by_zero_cb (TDouble *self, gpointer user_data) {
  g_print ("\nError: division by zero.\n\n");
}
\end{lstlisting}

\subsection{Signal connection}\label{signal-connection}

A signal and a handler are connected with the function
\href{https://docs.gtk.org/gobject/func.signal_connect.html}{\passthrough{\lstinline!g\_signal\_connect!}}.

\begin{lstlisting}[language=C]
g_signal_connect (self, "div-by-zero", G_CALLBACK (div_by_zero_cb), NULL);
\end{lstlisting}

\begin{itemize}
\tightlist
\item
  \passthrough{\lstinline!self!} is an instance the signal belongs to.
\item
  The second argument is the signal name.
\item
  The third argument is the signal handler. It must be casted by
  \passthrough{\lstinline!G\_CALLBACK!}.
\item
  The last argument is an user data. The signal doesn't need a user
  data, so NULL is assigned.
\end{itemize}

\subsection{Signal emission}\label{signal-emission}

Signals are emitted on the object. The following is a part of
\passthrough{\lstinline!tdouble.c!}.

\begin{lstlisting}[language=C]
TDouble *
t_double_div (TDouble *self, TDouble *other) {
... ... ...
  if ((! t_double_get_value (other, &value)))
    return NULL;
  else if (value == 0) {
    g_signal_emit (self, t_double_signal, 0);
    return NULL;
  }
  return t_double_new (self->value / value);
}
\end{lstlisting}

If the divisor is zero, the signal is emitted.
\href{https://docs.gtk.org/gobject/func.signal_emit.html}{\passthrough{\lstinline!g\_signal\_emit!}}
has three parameters.

\begin{itemize}
\tightlist
\item
  The first parameter is the instance that emits the signal.
\item
  The second parameter is the signal id. Signal id is the value returned
  by the function \passthrough{\lstinline!g\_signal\_new!}.
\item
  The third parameter is a detail. ``div-by-zero'' signal doesn't have a
  detail, so the argument is zero. Detail isn't explained in this
  section but usually you can put zero as a third argument. If you want
  to know the details, refer to
  \href{https://docs.gtk.org/gobject/concepts.html\#the-detail-argument}{GObject
  API Reference -- Signal Detail}.
\end{itemize}

If a signal has parameters, they are fourth and subsequent arguments.

\subsection{Example}\label{example}

A sample program is in src/tdouble3.

tdouble.h

\begin{lstlisting}[language=C, numbers=left]
#pragma once

#include <glib-object.h>

G_DECLARE_FINAL_TYPE (TDouble, t_double, T, DOUBLE, GObject)

/* getter and setter */
gboolean t_double_get_value (TDouble *self, double *value);
void     t_double_set_value (TDouble *self, double value);
TDouble *t_double_add (TDouble *self, TDouble *other);
TDouble *t_double_sub (TDouble *self, TDouble *other);
TDouble *t_double_mul (TDouble *self, TDouble *other);
TDouble *t_double_div (TDouble *self, TDouble *other);
TDouble *t_double_uminus (TDouble *self);
TDouble *t_double_new (double value);
\end{lstlisting}

tdouble.c

\begin{lstlisting}[language=C, numbers=left]
#include "tdouble.h"

#define T_TYPE_DOUBLE (t_double_get_type ())

static guint t_double_signal;

struct _TDouble
{
  GObject parent;
  double  value;
};

G_DEFINE_TYPE (TDouble, t_double, G_TYPE_OBJECT)

static void
t_double_class_init (TDoubleClass *class)
{
  GSignalFlags signal_flags = G_SIGNAL_RUN_LAST | G_SIGNAL_NO_RECURSE | G_SIGNAL_NO_HOOKS;
  t_double_signal
      = g_signal_new ("div-by-zero", G_TYPE_FROM_CLASS (class), signal_flags, 0, NULL, NULL, NULL, G_TYPE_NONE, 0);
}

static void
t_double_init (TDouble *self)
{
}

/* getter and setter */
gboolean
t_double_get_value (TDouble *self, double *value)
{
  g_return_val_if_fail (T_IS_DOUBLE (self), FALSE);

  *value = self->value;
  return TRUE;
}

void
t_double_set_value (TDouble *self, double value)
{
  g_return_if_fail (T_IS_DOUBLE (self));

  self->value = value;
}

/* arithmetic operator */
/* These operators create a new instance and return a pointer to it. */
#define t_double_binary_op(op)                                                                                         \
  if (!t_double_get_value (other, &value))                                                                             \
    return NULL;                                                                                                       \
  return t_double_new (self->value op value);

TDouble *
t_double_add (TDouble *self, TDouble *other)
{
  g_return_val_if_fail (T_IS_DOUBLE (self), NULL);
  g_return_val_if_fail (T_IS_DOUBLE (other), NULL);
  double value;

  t_double_binary_op (+)
}

TDouble *
t_double_sub (TDouble *self, TDouble *other)
{
  g_return_val_if_fail (T_IS_DOUBLE (self), NULL);
  g_return_val_if_fail (T_IS_DOUBLE (other), NULL);
  double value;

  t_double_binary_op (-)
}

TDouble *
t_double_mul (TDouble *self, TDouble *other)
{
  g_return_val_if_fail (T_IS_DOUBLE (self), NULL);
  g_return_val_if_fail (T_IS_DOUBLE (other), NULL);
  double value;

  t_double_binary_op (*)
}

TDouble *
t_double_div (TDouble *self, TDouble *other)
{
  g_return_val_if_fail (T_IS_DOUBLE (self), NULL);
  g_return_val_if_fail (T_IS_DOUBLE (other), NULL);
  double value;

  if ((!t_double_get_value (other, &value)))
    {
      return NULL;
    }
  else if (value == 0)
    {
      g_signal_emit (self, t_double_signal, 0);
      return NULL;
    }
  return t_double_new (self->value / value);
}

TDouble *
t_double_uminus (TDouble *self)
{
  g_return_val_if_fail (T_IS_DOUBLE (self), NULL);

  return t_double_new (-self->value);
}

TDouble *
t_double_new (double value)
{
  TDouble *d;

  d        = g_object_new (T_TYPE_DOUBLE, NULL);
  d->value = value;
  return d;
}
\end{lstlisting}

main.c

\begin{lstlisting}[language=C, numbers=left]
#include "tdouble.h"
#include <glib-object.h>

static void
div_by_zero_cb (TDouble *self, gpointer user_data)
{
  g_printerr ("\nError: division by zero.\n\n");
}

static void
t_print (char *op, TDouble *d1, TDouble *d2, TDouble *d3)
{
  double v1, v2, v3;

  if (!t_double_get_value (d1, &v1))
    {
      return;
    }
  if (!t_double_get_value (d2, &v2))
    {
      return;
    }
  if (!t_double_get_value (d3, &v3))
    {
      return;
    }

  g_print ("%lf %s %lf = %lf\n", v1, op, v2, v3);
}

int
main (void)
{
  TDouble *d1 = t_double_new (10.0);
  TDouble *d2 = t_double_new (20.0);
  TDouble *d3;
  if ((d3 = t_double_add (d1, d2)) != NULL)
    {
      t_print ("+", d1, d2, d3);
      g_object_unref (d3);
    }

  if ((d3 = t_double_sub (d1, d2)) != NULL)
    {
      t_print ("-", d1, d2, d3);
      g_object_unref (d3);
    }

  if ((d3 = t_double_mul (d1, d2)) != NULL)
    {
      t_print ("*", d1, d2, d3);
      g_object_unref (d3);
    }

  if ((d3 = t_double_div (d1, d2)) != NULL)
    {
      t_print ("/", d1, d2, d3);
      g_object_unref (d3);
    }

  g_signal_connect (d1, "div-by-zero", G_CALLBACK (div_by_zero_cb), NULL);
  t_double_set_value (d2, 0.0);
  if ((d3 = t_double_div (d1, d2)) != NULL)
    {
      t_print ("/", d1, d2, d3);
      g_object_unref (d3);
    }

  double v1, v3;

  if ((d3 = t_double_uminus (d1)) != NULL && (t_double_get_value (d1, &v1)) && (t_double_get_value (d3, &v3)))
    {
      g_print ("-%lf = %lf\n", v1, v3);
      g_object_unref (d3);
    }

  g_object_unref (d1);
  g_object_unref (d2);
}
\end{lstlisting}

Change your current directory to src/tdouble3 and type as follows.

\begin{lstlisting}
$ meson setup _build
$ ninja -C _build
\end{lstlisting}

Then, Executable file \passthrough{\lstinline!tdouble!} is created in
the \passthrough{\lstinline!\_build!} directory. Execute it.

\begin{lstlisting}
$ _build/tdouble
10.000000 + 20.000000 = 30.000000
10.000000 - 20.000000 = -10.000000
10.000000 * 20.000000 = 200.000000
10.000000 / 20.000000 = 0.500000

Error: division by zero.

-10.000000 = -10.000000
\end{lstlisting}

\subsection{Default signal handler}\label{default-signal-handler}

You may have thought that it was strange that the error message was set
in \passthrough{\lstinline!main.c!}. Indeed, the error happens in
\passthrough{\lstinline!tdouble.c!} so the message should been managed
by \passthrough{\lstinline!tdouble.c!} itself. GObject system has a
default signal handler that is set in the object itself. A default
signal handler is also called ``default handler'' or ``object method
handler''.

You can set a default handler with
\passthrough{\lstinline!g\_signal\_new\_class\_handler!}.

\begin{lstlisting}[language=C]
guint
g_signal_new_class_handler (const gchar *signal_name,
                            GType itype,
                            GSignalFlags signal_flags,
                            GCallback class_handler, /*default signal handler */
                            GSignalAccumulator accumulator,
                            gpointer accu_data,
                            GSignalCMarshaller c_marshaller,
                            GType return_type,
                            guint n_params,
                            ...);
\end{lstlisting}

The difference from \passthrough{\lstinline!g\_signal\_new!} is the
fourth parameter. \passthrough{\lstinline!g\_signal\_new!} sets a
default handler with the offset of the function pointer in the class
structure. If an object is derivable, it has its own class area, so you
can set a default handler with \passthrough{\lstinline!g\_signal\_new!}.
But a final type object doesn't have its own class area, so it's
impossible to set a default handler with
\passthrough{\lstinline!g\_signal\_new!}. That's the reason why we use
\passthrough{\lstinline!g\_signal\_new\_class\_handler!}.

The C file \passthrough{\lstinline!tdouble.c!} is changed like this. The
function \passthrough{\lstinline!div\_by\_zero\_default\_cb!} is added
and \passthrough{\lstinline!g\_signal\_new\_class\_handler!} replaces
\passthrough{\lstinline!g\_signal\_new!}. Default signal handler doesn't
have \passthrough{\lstinline!user\_data!} parameter. A
\passthrough{\lstinline!user\_data!} parameter is set in the
\passthrough{\lstinline!g\_signal\_connect!} family functions when a
user connects their own signal handler to the signal. Default signal
handler is managed by the instance, not a user. So no user data is given
as an argument.

\begin{lstlisting}[language=C, numbers=left]
static void
div_by_zero_default_cb (TDouble *self) {
  g_printerr ("\nError: division by zero.\n\n");
}

static void
t_double_class_init (TDoubleClass *class) {
  t_double_signal =
  g_signal_new_class_handler ("div-by-zero",
                              G_TYPE_FROM_CLASS (class),
                              G_SIGNAL_RUN_LAST | G_SIGNAL_NO_RECURSE | G_SIGNAL_NO_HOOKS,
                              G_CALLBACK (div_by_zero_default_cb),
                              NULL /* accumulator */,
                              NULL /* accumulator data */,
                              NULL /* C marshaller */,
                              G_TYPE_NONE /* return_type */,
                              0     /* n_params */
                              );
}
\end{lstlisting}

\passthrough{\lstinline!g\_signal\_connect!} and
\passthrough{\lstinline!div\_by\_zero\_cb!} are removed from
\passthrough{\lstinline!main.c!}.

Compile and execute it.

\begin{lstlisting}
$ cd src/tdouble4; _build/tdouble
10.000000 + 20.000000 = 30.000000
10.000000 - 20.000000 = -10.000000
10.000000 * 20.000000 = 200.000000
10.000000 / 20.000000 = 0.500000

Error: division by zero.

-10.000000 = -10.000000
\end{lstlisting}

The source file is in the directory src/tdouble4.

If you want to connect your handler (user-provided handler) to the
signal, you can still use \passthrough{\lstinline!g\_signal\_connect!}.
Add the following in \passthrough{\lstinline!main.c!}.

\begin{lstlisting}[language=C]
static void
div_by_zero_cb (TDouble *self, gpointer user_data) {
  g_print ("\nError happens in main.c.\n");
}

int
main (int argc, char **argv) {
... ... ...
  g_signal_connect (d1, "div-by-zero", G_CALLBACK (div_by_zero_cb), NULL);
... ... ...
}
\end{lstlisting}

Then, both the user-provided handler and default handler are called when
the signal is emitted. Compile and execute it, then the following is
shown on your display.

\begin{lstlisting}
10.000000 + 20.000000 = 30.000000
10.000000 - 20.000000 = -10.000000
10.000000 * 20.000000 = 200.000000
10.000000 / 20.000000 = 0.500000

Error happens in main.c.

Error: division by zero.

-10.000000 = -10.000000
\end{lstlisting}

This tells us that the user-provided handler is called first, then the
default handler is called. If you want your handler called after the
default handler, then you can use
\passthrough{\lstinline!g\_signal\_connect\_after!}. Add the lines below
to \passthrough{\lstinline!main.c!} again.

\begin{lstlisting}[language=C]
static void
div_by_zero_after_cb (TDouble *self, gpointer user_data) {
  g_print ("\nError has happened in main.c and an error message has been displayed.\n");
}

int
main (int argc, char **argv) {
... ... ...
  g_signal_connect_after (d1, "div-by-zero", G_CALLBACK (div_by_zero_after_cb), NULL);
... ... ...
}
\end{lstlisting}

Compile and execute it, then:

\begin{lstlisting}
10.000000 + 20.000000 = 30.000000
10.000000 - 20.000000 = -10.000000
10.000000 * 20.000000 = 200.000000
10.000000 / 20.000000 = 0.500000

Error happens in main.c.

Error: division by zero.

Error has happened in main.c and an error message has been displayed.

-10.000000 = -10.000000
\end{lstlisting}

The source files are in src/tdouble5.

\subsection{Signal flag}\label{signal-flag}

The order that handlers are called is described in
\href{https://docs.gtk.org/gobject/concepts.html\#signal-emission}{GObject
API Reference -- Sigmal emission}.

The order depends on the signal flag which is set in
\passthrough{\lstinline!g\_signal\_new!} or
\passthrough{\lstinline!g\_signal\_new\_class\_handler!}. There are
three flags which relate to the order of handlers' invocation.

\begin{itemize}
\tightlist
\item
  \passthrough{\lstinline!G\_SIGNAL\_RUN\_FIRST!}: the default handler
  is called before any user provided handler.
\item
  \passthrough{\lstinline!G\_SIGNAL\_RUN\_LAST!}: the default handler is
  called after the normal user provided handler (not connected with
  \passthrough{\lstinline!g\_signal\_connect\_after!}).
\item
  \passthrough{\lstinline!G\_SIGNAL\_RUN\_CLEANUP!}: the default handler
  is called after any user provided handler.
\end{itemize}

\passthrough{\lstinline!G\_SIGNAL\_RUN\_LAST!} is the most appropriate
in many cases.

Other signal flags are described in
\href{https://docs.gtk.org/gobject/flags.SignalFlags.html}{GObject API
Reference}.
